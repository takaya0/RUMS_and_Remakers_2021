\documentclass[11pt, a4paper, dvipdfmx]{jsarticle}

\usepackage{amsmath}
\usepackage{amsthm}
\usepackage{color}
\usepackage{ascmac}
\usepackage{amsfonts}
\usepackage{mathrsfs}
\usepackage{amssymb}
\usepackage{graphicx}
\usepackage{fancybox}
\usepackage{enumerate}
\usepackage{verbatim}
\usepackage{subfigure}
\usepackage{proof}
\usepackage{listings}
\usepackage{otf}
\usepackage{algorithm}
\usepackage{algorithmic}
\usepackage{physics}
\usepackage{tikz}
\usepackage{mathtools}
\usepackage[dvipdfmx]{hyperref}
\usepackage{pxjahyper}


\theoremstyle{definition}

\newtheorem{Axiom+}{Axiom}[section]
\newtheorem{Definition+}[Axiom+]{Definition}
\newtheorem{Theorem+}[Axiom+]{Theorem}
\newtheorem{Proposition+}[Axiom+]{Proposition}
\newtheorem{Lemma+}[Axiom+]{Lemma}
\newtheorem{Example+}[Axiom+]{Example}
\newtheorem{Corollary+}[Axiom+]{Corollary}
\newtheorem{Claim+}[Axiom+]{Claim}
\newtheorem{Property+}[Axiom+]{Property}
\newtheorem{Attention+}[Axiom+]{Attention}
\newtheorem{Question+}[Axiom+]{Question}
\newtheorem{Problem+}[Axiom+]{Problem}
\newtheorem{Consideration+}[Axiom+]{Consideration}
\newtheorem{Remark+}[Axiom+]{Remark}

\newcommand{\N}{\mathbb{N}}
\newcommand{\Z}{\mathbb{Z}}
\newcommand{\Q}{\mathbb{Q}}
\newcommand{\R}{\mathbb{R}}
\newcommand{\C}{\mathbb{C}}
\newcommand{\p}{\partial}
\newcommand{\lan}{\langle}
\newcommand{\ran}{\rangle}
\newcommand{\pal}{\parallel}
\newcommand{\dip}{\displaystyle}
\newcommand{\e}{\varepsilon}
\newcommand{\dl}{\delta}
\newcommand{\pphi}{\varphi}
\newcommand{\ti}{\tilde}
\newcommand{\argmax}{\mathop{\rm arg~max}\limits}
\newcommand{\argmin}{\mathop{\rm arg~min}\limits}

\pagestyle{empty}

\title{Optimization Algorithms on Riemannian Manifolds and their Applications}
\author{やたか}
\date{}
\begin{document}
    
    \maketitle
    \section*{Abstract}
    現代社会には様々な問題があるが, それを解決する数学の一つが「最適化数学」である. 
    最適化数学では, 制約空間$M$が$\R^n$と等しい\textbf{制約なし最適化問題}と,
    $M$が$\R^n$の真部分集合となる\textbf{制約つき最適化問題}の2つの問題を扱うが, 今回の講演では
    制約つき最適化問題, 特に制約空間$M$が($\R^n$に埋め込まれた)リーマン多様体であるものを扱う.
    制約空間がリーマン多様体の時は, 目的関数$f$を制約空間$M$に制限することで, 勾配降下法などの
    制約なし最適化問題のためのアルゴリズムをリーマン多様体上に拡張したものを適用することができる. そこで, 本講演では,  
    実問題に出てくるリーマン多様体の紹介をした後に, 制約空間がリーマン多様体となっている最適化問題に対して
    リーマン多様体にアルゴリズムを拡張する方法$\cite{absil}$を紹介し, 実際にそれらを用いて数値計算をした結果を紹介をする. 
    \section*{前提知識}
    本講演では, 時間の都合上多様体に関するある程度の知識(可微分多様体の定義や接空間など)は仮定する. 
    しかし, 実際の最適化問題を解く時はどちらかというと, こちらの文献\cite{sato}でも紹介されているように線形代数学や微分積分学の知識を使用するため, 
    1, 2回生でもある程度は楽しめる内容にはなっている.
    \begin{thebibliography}{9}
        \bibitem{absil} P.-A. ABSIL, R. MAHONY, AND R. SEPULCHRE, Optimization Algorithms on Matrix Manifolds, 
        Princeton University Press, Princeton, 2008.
        \bibitem{sato} 佐藤 寛之, 曲がった空間での最適化, 日本オペレーションズ・リサーチ学会, オペレーションズ・リサーチ Vol.60 9月号, 2015.
    \end{thebibliography}
    なお, 参考文献に関しては今後増える可能性があるため, 講演の際に改めて紹介する. 
\end{document}