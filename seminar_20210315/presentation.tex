\documentclass[dvipdfmx,11pt]{beamer}		% for my notebook computer and my Mac computer
%\documentclass[11pt]{beamer}			% for overleaf

\usepackage{amsmath}
\usepackage{amssymb}
%\usepackage{amsthm}
\usepackage{multicol}
\usepackage{listings}
\usepackage{otf}
\usepackage{enumerate}
\usepackage{algorithm}
\usepackage{algorithmic}
\usepackage{tikz}
\usepackage{mathtools}
\usepackage{comment}
\usetheme{Berlin}	%全体のデザイン

\useoutertheme[subsection=false]{smoothbars}	%デザインのカスタマイズ

\setbeamertemplate{navigation symbols}{}	%右下のちっちゃいナビゲーション記号を非表示

\AtBeginSubsection[]	%サブセクションごとに目次を表示
{\begin{frame}{Contents}
    \tableofcontents[currentsubsection]
\end{frame}}
\newtheorem{defi}{Definition}
\newtheorem{thm}[defi]{Theorem}
\newtheorem{prop}[defi]{Proposition}
\newtheorem{conj}[defi]{Conjecture}
\newtheorem{exam}[defi]{Example}
\newtheorem{prob}[defi]{Problem}
\newtheorem{set}[defi]{Setting}
\newtheorem{claim}[defi]{Claim}
\newtheorem{cor}[defi]{Corollary}
\newcommand{\N}{\mathbb{N}}
\newcommand{\R}{\mathbb{R}}
\newcommand{\Z}{\mathbb{Z}}
\newcommand{\X}{\mathcal{X}}
\newcommand{\Y}{\mathcal{Y}}
\newcommand{\Hil}{\mathcal{H}}
\newcommand{\Loss}{\mathcal{L}_{\mathbb{D}}}
\newcommand{\MLsp}{(\X, \Y, \mathbb{D}, \Hil, \Loss)}
\newcommand{\GAN}{(\X, \Y, \mathbb{D}, \Hil_{G}\times\Hil_{D}, \Loss)}
\newcommand{\argmax}{\mathop{\rm arg~max}\limits}
\newcommand{\argmin}{\mathop{\rm arg~min}\limits}

%---------------数理最適化--------------
\newcounter{mpproblem}[section]
\renewcommand{\thempproblem}{\thesection.\arabic{mpproblem}}
\makeatletter
\newenvironment{mpproblem}[1]%
{%
    \protected@edef\@currentlabelname{#1}%
    \par\vspace{\baselineskip}\noindent%
    \ifx#1\empty %
    \else \refstepcounter{mpproblem}$($#1$)$ %
    \fi%
    \hfill%
    $\left|%
    \hfill%
    \hspace{0.00\textwidth}%
    \@fleqntrue\@mathmargin\parindent%
    \begin{minipage}{0.86\textwidth}%
    \vspace{-1.0\baselineskip}%
}%
{%
    \end{minipage}%
    \@fleqnfalse%
    \right.$%
    \par\vspace{\baselineskip}\noindent%
    \ignorespacesafterend%
}%
\makeatother
\newcommand{\mpprobref}[1]{$($\nameref{#1}$)$}

\newenvironment{mpproblem*}%
{%
    \begin{mpproblem}{}%
}%
{%
    \end{mpproblem}%
    \ignorespacesafterend%
}
%-----------------------------------

\title{Optimization on Riemannian Manifolds}
\author{Takaya KOIZUMI}
\institute{Department of Mathematical Science\\B4}
\date{2021年1月28日 (木曜日)\\ 17:05~17:25}
\begin{document}
    \begin{frame}\frametitle{}
        \titlepage
    \end{frame}
    \section*{Contents}
    \begin{frame}\frametitle{Contents}
        \tableofcontents
    \end{frame}
\section{About Optimization}
    \subsection{Optimization Problem}
    \begin{frame}
        \frametitle{最適化問題とは}
        \begin{block}{最適化問題}
            $M$を$\R^n$の部分集合とする. 関数$f : \R^n\to\R$を$M$に制限した時に
            最大となる点$x^*\in M$を求める問題を最大化問題, 最小となる点$x_*\in M$
            を求める問題を最小化問題という. 最大化問題・最小化問題を合わせて(連続)最適化問題(Optimization Problem)といい, 
            この時の$f$を目的関数(objective function), $M$を制約空間(Constraint space)と呼ぶ. 
        \end{block}
        特に, $M = \R^n$の時, \textbf{制約なし}最適化問題といい, $M\neq\R^n$の時, \textbf{制約つき}最適化問題
        という. また, $f$を$\R^n$上ではなく有限集合$V\subset\Z$上の関数の時の最適化問題を 
        (離散)最適化問題と呼ぶ. 
    \end{frame}
    \begin{frame}
        \frametitle{最適化問題の具体例}
        \begin{exampleblock}{2次関数の最小化(制約なし)}
            $f : \R\to\R$を$f(x) = x^2 + 25x + 12$とする. 
            この$f$が最小となるような点$x^*\in\R$を求めよ.
        \end{exampleblock}
        \begin{exampleblock}{長方形の面積最大化(制約あり)}
            $f : \R^2\to\R$を$f(x, y) = xy$とする. $L\geq 0$とした時, 
            この$f$が最大となるような点$(x^*, y^*)\in M = \{(x, y)\mid x\geq 0, y\geq0, 2(x + y) = L\}$を求めよ.
        \end{exampleblock}        
    \end{frame}
    \begin{frame}
        \frametitle{最適化数学が活躍する領域}
        現代社会において, 様々な分野で最適化は必要とされている.
        \begin{exampleblock}{Delivery Scope 問題}
            Covid-19の影響で, 店舗配達の需要が増加し, 店舗の配達可能区域を決める必要性が高まった.
            できるだけ需要を高くした上で, 制限時間の間にお客に商品を届けることができる領域を求める問題.
        \end{exampleblock}
        \begin{exampleblock}{高次元目的関数最適化問題}
            データが簡単に収集しやすくなり, 高次元(1000から何千億次元)で複雑な目的関数
            の最適化の需要が高まっている(特に, 機械学習). 
        \end{exampleblock} 
    \end{frame}
    \subsection{Optimization Algorithm}
    \begin{frame}
        \frametitle{最適化アルゴリズム}
        最適化問題を解くためにほぼ必須となるのが, \textbf{最適化アルゴリズム} (Optimization Algorithm)である.
        これにより, 最適化問題を計算機に効率的に解かせることができる. 例えば, 目的関数が微分可能で, 制約がない最適化問題であれば
        \textbf{勾配降下法}(Gradient Decent) を使って解くこと(局所的最適解を求める)ができる.
        \begin{algorithm}[H]
            \caption{Gradient Decent(GD)}
            \begin{algorithmic}
                \REQUIRE $f$: differentiable function on $\R^{d}$
                \REQUIRE $0< t <1$ : 
                \STATE $x\leftarrow x_{0}\in\R^n$
                \WHILE{$x$ not converged} 
                \STATE $x\leftarrow x - t~\mathrm{grad} f(x)$ 
                \ENDWHILE
                \RETURN $x$
            \end{algorithmic}
        \end{algorithm}
    \end{frame}
    \begin{frame}
        \frametitle{制約付き最適化問題とリーマン多様体}
        GD法は便利な手法だが, $x\in M$であったとしても, $x - t~\mathrm{grad} f(x)\in M$
        となるとは限らないため, GD法を中心とした制約なし最適化問題用のアルゴリズムを制約付き最適化問題
        に用いることはできない. しかしながら, $M$がリーマン多様体である時は, $f$の定義域を$M$に制限することにより
        制約なしの最適化問題とみなすことで, これらのアルゴリズムをリーマン多様体に拡張したものを適用することができる. 
    \end{frame}
\section{Riemannian Manifolds}
    \subsection{Riemannian Manifold and Gradient}
    \begin{frame}
        \frametitle{Riemannian Manifold}
        \begin{defi}[Riemannian Manifold]
            $M$を可微分多様体とする. 任意の$x\in M$の接空間$T_{x}M$に
            内積$g : T_{x}M\times T_{x}M\to\R$が定まっている時, 組$(M, g)$を
            リーマン多様体(Riemannian Manifold)といい, $g$をリーマン計量(Riemannian metric)と呼ぶ. 
        \end{defi}
        \begin{defi}[Gradient]
            $(M, g)$をリーマン多様体とし, $f:M\to\R$を可微分写像とする. $x\in M$について
            \footnotesize
            \begin{align*}
                \forall\xi\in T_{x}M, g(\mathrm{grad} f(x), \xi) = Df(x)[\xi]
            \end{align*}
            \normalsize
            を満たす一意な$\mathrm{grad} f(x)\in T_{x}M$を$f$の$x\in M$での勾配(Gradient)
            という. 
        \end{defi}
    \end{frame}
    \subsection{Sphere}
    \begin{frame}
        \frametitle{具体例その1}
        \begin{exampleblock}{Sphere}
            自然数$n\geq2$について球面$S^{n - 1} \coloneqq\{x\in\R^n\mid x^{\top}x = 1\}$
            は$\R^n$に埋め込まれた可微分多様体である. また, $x\in S^{n - 1}$での接空間$T_{x}S^{n - 1}$
            は,
            \begin{align*}
                T_{x} S^{n - 1} = \{z\in\R^n\mid x^{\top}z = 0\}
            \end{align*}
            となる. $g_{x} : T_{x}S^{n - 1}\times T_{x}S^{n - 1}\to\R$を
            \begin{align*}
                g_{x}(\xi, \eta) = \xi^{\top}\eta
            \end{align*}
            と定めれば, $g_{x}$は内積となるので, $S^{n - 1}$はリーマン多様体である.
        \end{exampleblock}
    \end{frame}
    \subsection{Stiefel Manifold}
    \begin{frame}
        \frametitle{具体例その2}
        \begin{defi}[Stiefel Manifold]
            $X = [x_1 x_2 \ldots x_p]\in\R^{n\times p} (n\geq p)$で, 
            $\{x_{i}\}_{i = 1}^{n}$が正規直交系であるような$n\times p$行列全体は$\R^{n\times p}$に
            埋め込まれた可微分多様体となる. この多様体を
            シュティーフェル多様体(Stiefel Manifold)といい, $\operatorname{St}(p, n)$と表す. すなわち, 
            \begin{align*}
                \operatorname{St}(p, n) = \left\{X\in\R^{n\times p}\mid X^{\top}X = \mathbb{I}_{p}\right\}.
            \end{align*}
            ただし, $\mathbb{I}_{p}$は$p\times p$の単位行列である.
        \end{defi}
        $p = 1$の時, $\operatorname{St}(1, n) = S^{n - 1}$であり, $n = p$の時, $\operatorname{St}(n, n)$は$n$次の直交群$O(n)$となる.
    \end{frame}
    \begin{frame}
        \frametitle{シュティーフェル多様体の接空間}
        $X\in\operatorname{St}(p, n)$での接空間$T_{X}St(p, n)$は, 
        \begin{align*}
            T_{X} \operatorname{St}(p, n)&=\left\{Z \in \mathbb{R}^{n \times p}\mid X^{T} Z+Z^{T} X=0\right\}\\
                                         &= \left\{X \Omega+X_{\perp} K\mid \Omega\in\operatorname{Skew}(p), K \in \mathbb{R}^{(n-p) \times p}\right\}
        \end{align*}
        となる. ここで$\operatorname{Skew}(p) = \{\Omega\in\R^{p\times p}\mid \Omega^{\top} = -\Omega\}$であり, 
        $X_{\perp}$は$\operatorname{span}(X)^{\perp} = \operatorname{span}(X_{\perp})$を満たす$n\times (n - p)$行列である.
        $g_{X} : T_{X}\operatorname{St}(p, n)\times T_{X}\operatorname{St}(p, n)\to\R$を,
        \begin{align*}
            g_{X}(\xi, \eta) = \operatorname{tr}(\xi^{\top} \eta)
        \end{align*}
        と定めれば, $g_{X}$は内積となるので, $\operatorname{St}(p, n)$はリーマン多様体である. 
    \end{frame}
    \subsection{Grassmann Manifold}
    \begin{frame}
        \frametitle{具体例その3}
        $\R^{n\times p}_* = \{X\in\R^{n\times p}\mid \operatorname{rank}(X) = p\}$とする. 
        \begin{defi}[Grassmann Manifold]
            $\R^{n\times p}_{*}$上に二項関係$\sim$を
            \begin{align*}
                X\sim Y \iff \operatorname{span}(X) = \operatorname{span}(Y)
            \end{align*}
            と定義すると, $\sim$は同値関係になる. この時, 商空間$\R^{n\times p }_{*}/\sim$は, 
            可微分多様体となる. この多様体を, グラスマン多様体(Grassmann Manifold)といい, $\operatorname{Grass}(p, n)$と
            表す. 
        \end{defi}
        $p = 1$の時, $\operatorname{Grass}(1, n)$は射影空間$\R\mathbb{P}^{n - 1}$である. 
    \end{frame}
    \begin{frame}
        \frametitle{}
        \begin{prop}[基底の変換と特殊線形群]
            $X, Y\in\R^{n\times p}_{*}$について, 
            \begin{align*}
                \operatorname{span}(X) = \operatorname{span}(Y)\iff \exists M\in \operatorname{GL}_p(\R)\text{ s.t. } Y = XM
            \end{align*}
            が成立する. 
        \end{prop}
        これによって$\operatorname{Grass}(p, n) = R^{n\times p}_{*}/\operatorname{GL}_p(\R)$となる. 
        また, $g_{\operatorname{span}(X)} : T_{\operatorname{span}(X)}\operatorname{Grass}(p, n)\times T_{\operatorname{span}(X)}\operatorname{Grass}(p, n)\to\R$を
        \begin{align*}
            g_{\operatorname{span}(X)}(\xi, \eta) = \operatorname{tr}\left(\left(X^{T} X\right)^{-1} \overline{\xi}^{T} \overline{\eta}\right)
        \end{align*}
        と定義する. ただし, $\overline{\xi}, \overline{\eta}$はそれぞれ, $\xi, \eta$のhorizontal liftである.
        この時, $g_{\operatorname{span}(X)}$は代表元のとり方によらず内積の定義を満たすので, $\operatorname{Grass}(p, n)$はリーマン多様体となる.
    \end{frame}
\section{Optimization on Sphere}
    \subsection{Minimization of quadratic form}
    \begin{frame}
        \frametitle{行列の最大・最小固有値問題とリーマン多様体}
        リーマン多様体上の最適化の具体例として, 二次形式の最適化をすることで, 対称行列の
        最大・最小固有値を求める問題を扱う. そのために, まず二次形式の導入を行い, 重要な性質を
        紹介する. 
        \begin{defi}[二次形式]
            $A\in\R^{n\times n}$を対称行列とする. 関数$q_{A}:\R^{n}\to\R$を
            \begin{align*}
                q_{A}(x) = x^{\top}Ax
            \end{align*}
            と定義する. この時, この関数$q_{A}$を二次形式(quadratic form)と呼ぶ. 
        \end{defi}
    \end{frame}
    \begin{frame}
        \frametitle{二次形式と最大固有値・最小固有値}
        二次形式の重要な性質として, 以下の性質が知られている.
        \begin{prop}[二次形式と最大固有値・最小固有値\cite{qform}]
            $A\in\R^{n\times n}$を対称行列, $\lambda_{1}\leq\lambda_{2}\leq\cdots\leq\lambda_{n}$
            を$A$の固有値とする. この時, 
            \begin{enumerate}
                \item $\forall x\in S^{n - 1}$, $\lambda_{1}\leq q_{A}(x)\leq\lambda_{n}$
                \item $q_{A}(x_*) = \lambda_1$となるのは, $x_*\in S^{n - 1}$が最小値$\lambda_1$の固有ベクトルであるのみである.
                \item $q_{A}(x^*) = \lambda_n$となるのは, $x^*\in S^{n - 1}$が最大値$\lambda_n$の固有ベクトルであるのみである.
            \end{enumerate}
            が成立する. 
        \end{prop}
    \end{frame}
    \begin{frame}
        \frametitle{Optimization Problem on Sphere}
        先の性質に基づき、以下の球面上の最小化問題を考える.
        \begin{exampleblock}{球面上の二次形式最小化問題}
            $A\in\R^{n\times n}$を対称行列とし, $q_{A}$を二次形式とする. 
            \begin{mpproblem*}
                \begin{alignat*}{2}
                 &\text{minimize}   & \quad q_{A}(x) = x^{\top}Ax  \\
                 &\text{subject to} & \quad x\in S^{n - 1}  
                \end{alignat*}
            \end{mpproblem*}
        \end{exampleblock}
        また, $q_{A}$を$-q_{A}$とすれば最大固有値も求めることもできる. 
    \end{frame}
    \subsection{Riemannian Gradient Decent on Sphere}
    \begin{frame}
        \frametitle{勾配降下法の問題点}
        先の最適化問題を解くために, ユークリッド空間上の勾配降下法を球面上に拡張する. 
        まず, ユークリッド勾配$\mathrm{grad}_{\R^n}f(x)$を, 球面上の勾配
        $\mathrm{grad}f(x)$に置き換えると, 以下のようになる. 
        \begin{enumerate}
            \item 初期値$x_{0}\in S^{n - 1}$を決める.
            \item $x\leftarrow x - t~\mathrm{grad} f(x)$として$x$を更新する.
            \item 2を収束するまで繰り返す. 
        \end{enumerate}
        しかし, $x - t~\mathrm{grad} f(x)\notin S^{n - 1}$であるため, $\|x - t~\mathrm{grad} f(x)\|$
        で割り正規化することで, 更新後の点を$S^{n - 1}$となるようにする. 
    \end{frame}
    \begin{frame}
        \frametitle{球面上の勾配降下法}
        前スライドの議論を踏まえた上で, 球面上の勾配降下法を以下のように設計する. 
        \begin{algorithm}[H]
            \caption{Riemannian Gradient Decent on Sphere}
            \begin{algorithmic}
                \REQUIRE $f$: differentiable function on $S^{n - 1}$
                \REQUIRE $0< t <1$ : 
                \STATE $x\leftarrow x_{0}\in S^{n - 1}$
                \WHILE{$x$ not converged} 
                \STATE $d = -t~\mathrm{grad} f(x)$
                \STATE $x\leftarrow (x + d)/\|x + d\|$ 
                \ENDWHILE
                \RETURN $x$
            \end{algorithmic}
        \end{algorithm}
    \end{frame}
    \begin{frame}
        \frametitle{球面上の勾配の求め方}
        \begin{prop}[球面上の勾配と直交射影]
            目的関数$f:\R^{n}\to\R$を$S^{n - 1}$に制限した関数を$\overline{f}:S^{n - 1}\to\R$
            とする. この時, $x\in S^{n -1}$について, $P_x : T_{x}\R^{n}\to T_{x}S^{n - 1}$を
            $T_{x}\R^n = \R^n$から閉部分空間$T_{x}S^{n - 1}$への直交射影(orthogonal projection)とすれば,
            \begin{align*}
                \mathrm{grad} \overline{f}(x) = P_{x}(\mathrm{grad}_{\R^{n}}f(x))
            \end{align*}
            が成立する. また, 直交射影$P_x$は 
            \begin{align*}
                P_{x}(\xi) = (\mathbb{I}_{n} - xx^{\top})\xi
            \end{align*}
            となる. 
        \end{prop}
    \end{frame}
    \subsection{Experiments}
    \begin{frame}
        \frametitle{計算機実験}
        実際に具体的な対称行列$A\in\R^{2\times 2}$を考え, 実際にその最小固有値に
        収束することを計算機を用いて確認する\cite{sato}. 
        \begin{exampleblock}{}
            対称行列$A\in\R^{2\times 2}$を
            \begin{align*}
                A = \left(
                    \begin{array}{ccc}
                            2 & 2 \\
                            2 & 5 
                    \end{array}
                    \right)
            \end{align*}
            とする. この時, $A$の固有値は$\lambda_{1} = 1, \lambda_{2} = 6$
            である. また, $q_{A}(x, y) = 2x^2 + 4xy + 5y^2$より, 
            \begin{align*}
                \mathrm{grad}_{\R^2}f(x) = (4(x + y), 4x + 10y)^{\top}
            \end{align*}
            である. 
        \end{exampleblock}
    \end{frame}
    \begin{frame}
        \frametitle{実験結果(最小固有値)}
        $q_{A}$の最小値$\lambda_1 = 1$に収束していることが確かめられた. 
        また, その時の$x$は$\lambda_1$の固有ベクトル$x_{\lambda_1} = (2/\sqrt{5}, -1/\sqrt{5})$となっている. なお, $t = 0.01$, $x_{0} = (1, 0)$とした. 
        \begin{figure}[b]
            \begin{tabular}{c}
                \begin{minipage}{0.55\hsize}
                    \centering
                    \includegraphics[width = 6.3cm]{result.png}
                \end{minipage}
                \begin{minipage}{0.45\hsize}
                    \centering
                    \includegraphics[width = 5.5cm]{eign_vec_min.png}
                \end{minipage}
            \end{tabular}   
        \end{figure}  
    \end{frame}
    \begin{frame}
        \frametitle{実験結果(最大固有値)}
        $q_{A}$の最大値$\lambda_2 = 6$に収束していることが確かめられた.
        また,その時の$x$は$\lambda_2$の固有ベクトル$x_{\lambda_2} = (1/\sqrt{5}, 2/\sqrt{5})$となっている.
        なお, $t = 0.01$, $x_{0} = (1, 0)$とした. 
        \begin{figure}[b]
            \begin{tabular}{c}
                \begin{minipage}{0.55\hsize}
                    \centering
                    \includegraphics[width = 6.3cm]{result_max.png}
                \end{minipage}
                \begin{minipage}{0.45\hsize}
                    \centering
                    \includegraphics[width = 5.5cm]{eign_vec_max.png}
                \end{minipage}
            \end{tabular}   
        \end{figure}  
    \end{frame}
\section*{References}
    \begin{frame}\frametitle{References}
        \begin{thebibliography}{9}
            \beamertemplatetextbibitems
            \bibitem{} P.-A. ABSIL, R. MAHONY, AND R. SEPULCHRE, Optimization Algorithms on Matrix Manifolds, 
                       Princeton University Press, Princeton, NJ, 2008.
            \bibitem{qform} https://www.cck.dendai.ac.jp/math/support/advance/2次形式.pdf, 東京電機大学
            \bibitem{sato} 佐藤 寛之, 曲がった空間での最適化, 日本オペレーションズ・リサーチ学会, オペレーションズ・リサーチ Vol.60 9月号, 2015
	    \end{thebibliography}
    \end{frame}
\end{document}


