%=====================================
%    この tex ファイルは『ふきよせ』
%    の記事作成テンプレートです. 
%=====================================

% -----------------------
% preamble
% -----------------------
% ここから本文 (\begin{document}) までの
% ソースコードに変更を加えた場合は
% 編集者まで連絡してください. 
% Don't change preamble code yourself. 
% If you add something
% (usepackage, newtheorem, newcommand, renewcommand),
% please tell it 
% to the editor of institutional paper of RUMS.

% ------------------------
% documentclass
% ------------------------
\documentclass[11pt, a4paper, dvipdfmx]{jsarticle}

% ------------------------
% usepackage
% ------------------------
\usepackage{algorithm}
\usepackage{algorithmic}
\usepackage{amscd}
\usepackage{amsfonts}
\usepackage{amsmath}
\usepackage[psamsfonts]{amssymb}
\usepackage{amsthm}
\usepackage{ascmac}
\usepackage{color}
\usepackage{enumerate}
\usepackage{fancybox}
\usepackage[stable]{footmisc}
\usepackage[dvips]{graphicx}
\usepackage{listings}
\usepackage{mathrsfs}
\usepackage{mathtools}
\usepackage{otf}
%\usepackage{physics}
\usepackage{pifont}
\usepackage{proof}
\usepackage{subfigure}
\usepackage{thmbox}
\usepackage{tikz}
\usepackage{verbatim}
\usepackage[all]{xy}

\usetikzlibrary{cd}



% ================================
% パッケージを追加する場合のスペース 

%=================================


% --------------------------
% theoremstyle
% --------------------------
\theoremstyle{definition}

% --------------------------
% newtheoem
% --------------------------

% 日本語で定理, 命題, 証明などを番号付きで用いるためのコマンドです. 
% If you want to use theorem environment in Japanece, 
% you can use these code. 
% Attention!
% All theorem enivironment numbers depend on 
% only section numbers.
\theoremstyle{definition}
%
%%%%%%%%%%%%%%%%%%%%%%%%%%%%%%%%%%%%%%
%ここにないパッケージを入れる人は,必ずここに記載すること.
%
%%%%%%%%%%%%%%%%%%%%%%%%%%%%%%%%%%%%%%
%ここからはコード表です.
%
\newtheorem{Axiom}{公理}[section]
\newtheorem{Definition}{定義}[section]
\newtheorem{Theorem}{定理}[section]
\newtheorem{Proposition}[Theorem]{命題}
\newtheorem{Lemma}[Theorem]{補題}
\newtheorem{Corollary}[Theorem]{系}
\newtheorem{Example}{例}[section]
\newtheorem{Claim}{主張}[section]
\newtheorem{Property}{性質}[section]
\newtheorem{Attention}{注意}[section]
\newtheorem{Question}{問}[section]
\newtheorem{Problem}{問題}[section]
\newtheorem{Consideration}{考察}[section]
\newtheorem{Alert}{警告}[section]
%%%%%%%%%%%%%%%%%%%%%%%%%%%%%%%%%%%%%%
%
%定義や定理等に番号をつけたくない場合(例えば定理1.1等)は以下のコードを使ってください.
%但し,例えば\Axiom*{}としてしまうと番号が付いてしまうので,必ず \begin{Axiom*} \end{Axiom*}の形で使ってください.

%
%%%%%%%%%%%%%%%%%%%%%%%%%%%%%%%%%%%%%%
%英語で定義や定理を書きたい場合こっちのコードを使うこと.
\newtheorem{Axiom+}{Axiom}[section]
\newtheorem[S]{Defi}[Axiom+]{Definition}
\newtheorem[S]{Thm}[Axiom+]{Theorem}
\newtheorem[S]{Prop}[Axiom+]{Proposition}
\newtheorem[S]{Lem}[Axiom+]{Lemma}
\newtheorem[S]{Ex}[Axiom+]{Example}
\newtheorem[S]{Cor}[Axiom+]{Corollary}
\newtheorem[S]{Prf}[Axiom+]{Proof}
\newtheorem{Claim+}{Claim}
\newtheorem{Property+}{Property}
\newtheorem{Attention+}{Attention}
\newtheorem{Question+}{Question}
\newtheorem{Problem+}{Problem}
\newtheorem{Consideration+}{Consideration}
\newtheorem{Alert+}{Alert}
%
%
% ----------------------------
% commmand
% ----------------------------
% 執筆に便利なコマンド集です. 
% コマンドを追加する場合は下のスペースへ. 

% 集合の記号 (黒板文字)
\newcommand{\NN}{\mathbb{N}}
\newcommand{\ZZ}{\mathbb{Z}}
\newcommand{\QQ}{\mathbb{Q}}
\newcommand{\RR}{\mathbb{R}}
\newcommand{\CC}{\mathbb{C}}
\newcommand{\PP}{\mathbb{P}}
\newcommand{\KK}{\mathbb{K}}


% 集合の記号 (太文字)
\newcommand{\nn}{\mathbf{N}}
\newcommand{\zz}{\mathbf{Z}}
\newcommand{\qq}{\mathbf{Q}}
\newcommand{\rr}{\mathbf{R}}
\newcommand{\cc}{\mathbf{C}}
\newcommand{\pp}{\mathbf{P}}
\newcommand{\kk}{\mathbf{K}}

% 特殊な写像の記号
\newcommand{\ev}{\mathop{\mathrm{ev}}\nolimits} % 値写像
\newcommand{\pr}{\mathop{\mathrm{pr}}\nolimits} % 射影

% スクリプト体にするコマンド
%   例えば {\mcal C} のように用いる
\newcommand{\mcal}{\mathcal}

% 花文字にするコマンド 
%   例えば {\h C} のように用いる
\newcommand{\h}{\mathscr}

% ヒルベルト空間などの記号
\newcommand{\F}{\mcal{F}}
\newcommand{\X}{\mcal{X}}
\newcommand{\Y}{\mcal{Y}}
\newcommand{\Hil}{\mcal{H}}
\newcommand{\RKHS}{\Hil_{k}}
\newcommand{\Loss}{\mcal{L}_{D}}
\newcommand{\MLsp}{(\X, \Y, D, \Hil, \Loss)}

% 偏微分作用素の記号
\newcommand{\p}{\partial}

% 角カッコの記号 (内積は下にマクロがあります)
\newcommand{\lan}{\langle}
\newcommand{\ran}{\rangle}



% 圏の記号など
\newcommand{\Set}{{\bf Set}}
\newcommand{\Vect}{{\bf Vect}}
\newcommand{\FDVect}{{\bf FDVect}}
\newcommand{\Ring}{{\bf Ring}}
\newcommand{\Ab}{{\bf Ab}}
\newcommand{\Mod}{\mathop{\mathrm{Mod}}\nolimits}
\newcommand{\CGA}{{\bf CGA}}
\newcommand{\GVect}{{\bf GVect}}
\newcommand{\Lie}{{\bf Lie}}
\newcommand{\dLie}{{\bf Liec}}



% 射の集合など
\newcommand{\Map}{\mathop{\mathrm{Map}}\nolimits}
\newcommand{\Hom}{\mathop{\mathrm{Hom}}\nolimits}
\newcommand{\End}{\mathop{\mathrm{End}}\nolimits}
\newcommand{\Aut}{\mathop{\mathrm{Aut}}\nolimits}
\newcommand{\Mor}{\mathop{\mathrm{Mor}}\nolimits}

% その他便利なコマンド
\newcommand{\dip}{\displaystyle} % 本文中で数式モード
\newcommand{\e}{\varepsilon} % イプシロン
\newcommand{\dl}{\delta} % デルタ
\newcommand{\pphi}{\varphi} % ファイ
\newcommand{\ti}{\tilde} % チルダ
\newcommand{\pal}{\parallel} % 平行
\newcommand{\op}{{\rm op}} % 双対を取る記号
\newcommand{\lcm}{\mathop{\mathrm{lcm}}\nolimits} % 最小公倍数の記号
\newcommand{\Probsp}{(\Omega, \F, \P)} 
\newcommand{\argmax}{\mathop{\rm arg~max}\limits}
\newcommand{\argmin}{\mathop{\rm arg~min}\limits}





% ================================
% コマンドを追加する場合のスペース 

% =================================





% ---------------------------
% new definition macro
% ---------------------------
% 便利なマクロ集です

% 内積のマクロ
%   例えば \inner<\pphi | \psi> のように用いる
\def\inner<#1>{\langle #1 \rangle}

% ================================
% マクロを追加する場合のスペース 

%=================================





% ----------------------------
% documenet 
% ----------------------------
% 以下, 本文の執筆スペースです. 
% Your main code must be written between 
% begin document and end document.
% ---------------------------

\title{ふきよせテンプレート}
\author{筆者名}
\date{}
\begin{document}
\maketitle

ふきよせのテンプレートです. 
以下を書き換えて記事を作成してください. 
がんば!

\section{定理環境のサンプル}

\begin{Definition}
    定義定義定義定義定義
\end{Definition}

\begin{Defi}
    definition definition definition
\end{Defi}

\subsection*{コマンド集のサンプル}
プリアンブル(ソースコードの上の方に書いてあるコード)
のコマンドを使った場合のサンプルを書いておきます. 
\begin{itemize}
    \item $\NN \QQ \RR \CC \PP \KK$
    \item $\nn \qq \rr \cc \pp \kk$
    \item $\pr_1:X\times Y \to X$
    \item $\forall \e >0 \quad \exists \dl > 0 \quad 
          \forall x \in U_{\dl}(a) \quad |f(x) - f(a)| < \e$
    \item ${\mcal C} \simeq \Mod(R)$
    \item ${\h C}^\op \to \Set$
    \item $\inner<\pphi | \psi>$
\end{itemize}

%===============================================
% 参考文献スペース
%===============================================
\begin{thebibliography}{20} 
    \bibitem{ひ1} 筆者, 『本の名前』, 出版社, 出版年.
    \bibitem{AB1} A.\ Author, B.\ Buthor, \textit{Title of The Book}, Publisher, 9999.
\end{thebibliography}

%===============================================




\end{document}
